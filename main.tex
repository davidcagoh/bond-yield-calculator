\documentclass[10pt]{article}
\usepackage[utf8]{inputenc}
\usepackage[a4paper, margin=0.6in, top=0.45in, bottom=0.45in]{geometry}
\usepackage{amsmath, amsfonts, amssymb, amsthm, mathrsfs}
\usepackage{graphicx, booktabs, hyperref, subcaption, enumerate}
\usepackage{lmodern, xcolor}

\setlength{\parskip}{2pt}
\setlength{\itemsep}{1pt}
\setlength{\belowcaptionskip}{4pt}

\title{\vspace{-12pt}\textbf{MAT1856/APM466 Assignment \#1: The Value of Time}\vspace{-6pt}}
\author{David Goh \quad Student \#: 1012375243 \quad February 2, 2026\vspace{-8pt}}
\date{}
\begin{document}
\maketitle
\vspace{-10pt}

\section*{Fundamental Questions (25 pts)}
\begin{enumerate}
\item \textbf{Q1 (5 pts)}
\begin{enumerate}
\item Governments issue bonds to finance spending without causing inflation, whereas printing money increases the money supply and can devalue the currency.
\item If investors expect short-term rates to fall or anticipate slower growth, demand for long-term bonds rises and their yields fall, flattening the curve.
\item Quantitative easing is when a central bank buys government or mortgage-backed securities to inject liquidity; since COVID-19 the US Fed used QE by purchasing Treasury and mortgage-backed bonds to stabilize markets and support lending.
\end{enumerate}

\item \textbf{Q2 (10 pts)} Ten bonds (0.5--5\,y to maturity, 5 Jan 2026): Canadian government fixed-coupon, semiannual, from Markets Insider (Frankfurt listing), evenly spaced maturities, no near-duplicates.
\begin{center}
\footnotesize
\begin{tabular}{@{}llll@{}}
\toprule
CAN 4 Aug 26 & CAN 1 Sep 26 & CAN 3 Feb 27 & CAN 2.75 May 27 \\
CAN 2.5 Aug 27 & CAN 2.5 Nov 27 & CAN 2 Jun 28 & CAN 4 Mar 29 \\
CAN 2.75 Mar 30 & CAN 0.5 Dec 30 & & \\
\bottomrule
\end{tabular}
\end{center}

\item \textbf{Q3 (10 pts)} Eigenvalues give each mode's contribution to total variability; eigenvectors give the shape (e.g.\ parallel shift, tilt, curvature). The largest eigenvalue and its eigenvector identify the dominant movement of the curve over time.
\end{enumerate}

\section*{Empirical Questions (75 pts)}
\begin{enumerate}
\setcounter{enumi}{3}
\item \textbf{Q4}
\begin{enumerate}
\item \textbf{4(a) (10 pts)} YTM computed for each bond; 5-year yield curve per day superimposed. Interpolation: linear over 0.6--5\,y (from 0.545\,y); yields from dirty prices.
\item \textbf{4(b) (15 pts)} \textbf{Spot curve.} (1) Per date: sort bonds by maturity $T$. (2) First bond: $r = (CF/P)^{1/T}-1$. (3) For each next bond: discount prior cash flows with known spots; residual $= P - \mathrm{PV}(\mathrm{prev})$; $r = (CF_{\mathrm{last}}/\mathrm{residual})^{1/t_{\mathrm{last}}}-1$. (4) Interpolate to 1--5\,y; repeat for all dates.
\item \textbf{4(c) (15 pts)} \textbf{Forward curve.} (1) Per date: get spot curve; interpolate to 1--5\,y. (2) $S_{\mathrm{cont}} = 2\ln(1+S_{\mathrm{semi}}/2)$. (3) $F_{1,n} = (S_{1+n}(1+n) - S_1)/n$, $n=1,\ldots,4$. (4) Store and plot.
\end{enumerate}

\begin{figure}[!ht]
\centering
\begin{minipage}{0.48\textwidth}
\includegraphics[width=\linewidth]{figures/fig_yield_curve.png}
\subcaption{Yield curves (YTM) per date.}
\end{minipage}\hfill
\begin{minipage}{0.48\textwidth}
\includegraphics[width=\linewidth]{figures/fig_spot_curve.png}
\subcaption{Spot curves per date.}
\end{minipage}\\[4pt]
\begin{minipage}{0.48\textwidth}
\includegraphics[width=\linewidth]{figures/fig_forward_curve.png}
\subcaption{1-year forward curves per date.}
\end{minipage}\hfill
\begin{minipage}{0.48\textwidth}
\includegraphics[width=\linewidth]{figures/fig_covariance_spot_rates.png}
\subcaption{Covariance: spot rates.}
\end{minipage}\\[4pt]
\begin{minipage}{0.48\textwidth}
\includegraphics[width=\linewidth]{figures/fig_covariance_forward_rates.png}
\subcaption{Covariance: forward rates.}
\end{minipage}
\caption{Q4--Q5: Yield, spot, forward curves and covariance matrices.}
\label{fig:all}
\end{figure}

\item \textbf{Q5 (20 pts)} Covariance matrices for spot and forward rates: see Fig.~\ref{fig:all}(d)--(e).

\item \textbf{Q6 (15 pts)} PCA on spot and forward rates. All components (eigenvalues and eigenvectors):
\begin{table}[!ht]
\centering\footnotesize
\begin{tabular}{@{}lcc|cc@{}}
\toprule
& \multicolumn{2}{c}{Spot rates} & \multicolumn{2}{c}{Forward rates} \\
& Eigenval. & Expl.\ (\%) & Eigenval. & Expl.\ (\%) \\
\midrule
PC1 & $8.67{\times}10^{-5}$ & 59.24 & $1.17{\times}10^{-4}$ & 68.75 \\
PC2 & $4.89{\times}10^{-5}$ & 33.42 & $4.16{\times}10^{-5}$ & 24.37 \\
PC3 & $4.83{\times}10^{-6}$ & 3.30 & $7.28{\times}10^{-6}$ & 4.27 \\
PC4 & $3.17{\times}10^{-6}$ & 2.17 & $4.46{\times}10^{-6}$ & 2.61 \\
PC5 & $2.74{\times}10^{-6}$ & 1.87 & -- & -- \\
\bottomrule
\end{tabular}
\caption{PCA eigenvalues (all components).}
\end{table}

\begin{table}[!ht]
\centering\footnotesize
\begin{tabular}{@{}lccccc@{}}
\toprule
& M1 & M2 & M3 & M4 & M5 \\
\midrule
PC1 & $-0.358$ & $-0.548$ & $-0.120$ & $-0.727$ & $-0.170$ \\
PC2 & $-0.394$ & $0.220$ & $-0.445$ & $0.271$ & $-0.725$ \\
PC3 & $0.554$ & $0.369$ & $0.253$ & $-0.470$ & $-0.521$ \\
PC4 & $0.573$ & $-0.672$ & $-0.216$ & $0.321$ & $-0.263$ \\
PC5 & $0.285$ & $0.251$ & $-0.823$ & $-0.269$ & $0.325$ \\
\bottomrule
\end{tabular}
\quad
\begin{tabular}{@{}lcccc@{}}
\toprule
& F2 & F3 & F4 & F5 \\
\midrule
PC1 & $0.681$ & $-0.113$ & $0.708$ & $-0.146$ \\
PC2 & $-0.015$ & $0.474$ & $0.263$ & $0.840$ \\
PC3 & $-0.689$ & $-0.401$ & $0.603$ & $0.024$ \\
PC4 & $0.247$ & $-0.776$ & $-0.254$ & $0.522$ \\
\bottomrule
\end{tabular}
\caption{PCA eigenvectors (all PCs): spot (left), forward (right).}
\end{table}
\textbf{Interpretation:} PC1 captures $\sim$59\% (spot) and $\sim$69\% (forward) of variability; its eigenvector is a near-parallel shift (spot) and short/mid forwards moving together (forward).
\end{enumerate}

\vfill
\section*{References and GitHub}
\footnotesize
[1] Bond data scraper (Jaspreet Khela):\\
\url{https://colab.research.google.com/drive/1kCYtYmExgO7-iXjSc_2Pj87BsRBJGZnp?usp=sharing}

\textbf{GitHub:}
\url{https://github.com/davidcagoh/bond-yield-calculator}
\end{document}
